\subsection{Anti-Debugging}
\begin{frame}[fragile]
    \frametitle{Debugger Detection 1 of 4}
    \begin{itemize}
        \item Kernel32.IsDebuggerPresent()
            \pedbullet{Most easily discovered and defeatable}
        \item INT 3 (0xCC) scans and CRC checks (triggered by means of exceptions)
            \pedbullet{Easy to bypass with hardware breakpoints}
        \item Timers
            \pedbullet{Very basic but powerful}
            \pedbullet{Example:}
    \end{itemize}
    \begin{tiny}
        \begin{semiverbatim}
            start = GetTickCount();
            do_some\_stuff();
            if (GetTickCount() � start > threshold)
            debugger\_detected();
        \end{semiverbatim}
    \end{tiny}
    \begin{itemize}
        \item RDTSC � Read Time-Stamp Counter (EDX:EAX)
    \end{itemize}
\end{frame}


\begin{frame}[fragile]
    \frametitle{Debugger Detection 2 of 4}
    \begin{itemize}
        \item CheckRemoteDebuggerPresent()
            \pedbullet{Queries the debugger port by calling NtQueryInformationProcess()}
            \pedbullet{Harder to defeat but doable through hooks}
        \item Detecting hardware breakpoints
            \pedbullet{Install a SEH, trigger an exception and check the DR* registers in the process' context structure}
            \pedbullet{Can also set magic values and verify they are kept}
            \pedbullet{There are other ways of retrieving the process' context structure}
        \item INT 2Dh
            \pedbullet{Without a debugger running we can trigger an exception and catch it with a SEH}
            \pedbullet{With a debugger present, we won't normally get control after the exception}
            \pedbullet{Additionally execution will continue a byte ahead from the address immediately after the INT 2Dh}
    \end{itemize}
\end{frame}


\begin{frame}[fragile]
    \frametitle{Debugger Detection 3 of 4}
    \begin{itemize}
        \item PEB.BeingDebugged
            \pedbullet{We can get the value directly and compare it to what's returned by IsDebuggerPresent(), if it's different a debugger might be trying to trick us}
        \item NtGlobalFlag
            \pedbullet{PEB -$>$ NtGlobalFlag. If 70h == FLG\_HEAP\_ENABLE\_TAIL\_CHECK, FLG\_HEAP\_ENABLE\_FREE\_CHECK and FLG\_HEAP\_VALIDATE\_PARAMETERS, implies debugger is active}
        \item Query the debbuger port
            \pedbullet{NtQueryInformationProcess(-1, 7,\&dword\_var, 4, 0)}
            \pedbullet{Will return the debugger port is one is attached}
            \pedbullet{Tricky to work around, we can either hook ZwQueryInformationProcess or intercept the subsequent syscall with a driver}
    \end{itemize}
\end{frame}

\begin{frame}[fragile]
    \frametitle{Debugger Detection 4 of 4}
    \begin{itemize}
        \item PEB.ProcessHeap-$>$ForceFlags
            \pedbullet{If not equal to zero implies a debugger is active, but only if the process was started by the debugger}
        \item Memory tags
            \pedbullet{If a process is started for debugging, Windows will tag freed and reserved memory by filling it with 0xFEEEFEEE}
            \pedbullet{PEB.Ldr points to \_PEB\_LDR\_DATA, which is a good candidate where to scan for the pattern}
        \item Spotting Single-Stepping
            \pedbullet{Set SEH and set the Trap Flag:}
            \begin{tiny}
                \begin{semiverbatim}
                    PUSHFD
                    XOR DWORD PTR[ESP],154h
                    POPFD
                \end{semiverbatim}
            \end{tiny}
            \pedbullet{or execute INT1 (0xF1) which will generate a Single Step exception}
            \pedbullet{If the SEH is called, no debugger is attached}
    \end{itemize}
\end{frame}


%\begin{frame}[fragile]
%    \frametitle{Debugger Detection}
%    \begin{itemize}
%        \item item
%            \pedbullet{description}
%    \end{itemize}
%    \begin{tiny}
%        \begin{semiverbatim}
%            code
%        \end{semiverbatim}
%    \end{tiny}
%\end{frame}


\begin{frame}
    \frametitle{Debugger Pre-Interaction Execution}
    \begin{itemize}
        \item When loading a target malicious binary do not assume that malicious code can not execute prior to the initial break
        \item DLLMain() execution occurs pre-interaction (unless you hook DLL load events)
            \pedbullet{http://www.security-assessment.com/Whitepapers/PreDebug.pdf}
        \item This technique has been known by the �underground� for some time
        \item PE Thread Local Storage (TLS) initialization startup code
    \end{itemize}
\end{frame}


\begin{frame}
    \frametitle{A Look Into the TLS Trick}
    \begin{itemize}
        \item \emph{TLS} stands for \emph{Thread Local Storage} and it's meant to be used to allocate storage for thread-specific data
        \item The \emph{TLS} structure, \emph{IMAGE\_TLS\_DIRECTORY}, pointed to by the \emph{TLS} directory entry has a small number of fields
        \item The one of special interest is \emph{AddressOfCallBacks}, pointing to a list of callbacks 
        \item TLS callbacks have the following form:
            \pedbullet{typedef void (MODENTRY *PIMAGE\_TLS\_CALLBACK) ( PTR DllHandle, UINT32 Reason, PTR Reserved );}
    \end{itemize}
\end{frame}


\begin{frame}
    \frametitle{Finding the TLS Structure}
        \begin{center}
            \pgfimage[height=7cm]{images/anti_re/tls_construction_a}
        \end{center}
\end{frame}


\begin{frame}
    \frametitle{The TLS Structure}
        \begin{center}
            \pgfimage[width=11cm]{images/anti_re/tls_construction_b}
        \end{center}
\end{frame}


\begin{frame}
    \frametitle{TLS Trick Conclusions}
    \begin{itemize}
        \item Inserting code as a \emph{TLS} callback allows to run our code before the main entry point of the program is reached
        \item This could be used to decrypt or otherwise modify the image of the file in a way that confuses someone trying to debug it and is unaware of this functionality
        \item IDA is clever and knows this, it will find the TLS entry point and offer it as the starting point in the disassembly
        \item For more on this, check out the blog post at
            \pedbullet{http://blog.dkbza.org/2007/03/pe-trick-thread-local-storage.html}
    \end{itemize}
\end{frame}



\begin{frame}[fragile]
    \frametitle{OllyDbg Vulnerability}
    \begin{itemize}
        \item Format string vulnerability in OutputDebugString()
            \pedbullet{http://www.securiteam.com/windowsntfocus/5ZP0N00DFE.html}
            \pedbullet{To reproduce, attach to an instantiation of Python:}
    \end{itemize}
    \begin{tiny}
    \begin{semiverbatim}
            import win32api
            win32api.OutputDebugString("%s"    * 50)       # crash
            win32api.OutputDebugString("%8.8x" * 15)       # stack data
    \end{semiverbatim}
    \end{tiny}
    \begin{itemize}
        \item Format string vulnerability in INT 3 processing
            \pedbullet{Exploitable when module name contains format string token}
            \pedbullet{http://www.securiteam.com/windowsntfocus/5WP0B1FFPA.html}
        \item Latest version, v1.10d still vulnerable!
            \pedbullet{The former is easier to exploit than the latter}
    \end{itemize}
\end{frame}


\subsection{Anti-Disassembling}
\begin{frame}
    \frametitle{Disassembler Mucking}
    \begin{itemize}
        \item JMP-ing or CALL-ing into an instruction (ASPack)
            \pedbullet{Breaks disassembly}
        \item Executable packing, crypting or otherwise encoding
        \item PE header modifications
            \pedbullet{As shown before}
        \item Advanced compiler optimizations
        \item Vulnerabilities ;-)
    \end{itemize}
\end{frame}


\begin{frame}
    \frametitle{Advanced Compiler Optimizations}
    \begin{itemize}
        \item \emph{PGO} (\emph{Profiling Guided Optimization}) produces multi-chunked functions
        \item Problems arise when those chunks are shared by different functions (only the tail of the function)
        \item For more details refer to:
            \pedbullet{http://blog.dkbza.org/2006/12/simply-blocks-basically.html}
            \pedbullet{http://blog.dkbza.org/2007/01/binnavis-basic-block-handling.html}
    \end{itemize}
\end{frame}

\begin{frame}
    \frametitle{Basic Block Rearranging Illustrated. Unoptimized layout}
        \begin{center}
            \pgfimage[height=7cm]{images/anti_re/basic_block_sharing_optimizations_1a}
        \end{center}
\end{frame}

\begin{frame}
    \frametitle{Basic Block Rearranging Illustrated. Optimized layout}
        \begin{center}
            \pgfimage[height=6cm]{images/anti_re/basic_block_sharing_optimizations_1b}
        \end{center}
\end{frame}

\begin{frame}
    \frametitle{Basic Block Sharing Problem}
    \begin{itemize}
        \item Counter-intuitively, the same code can belong to two, otherwise logically different basic blocks.
        \item This is a side-effect of the basic-block sharing, caused when one of the functions has a branch targeting the shared code, while the other doesn't
    \end{itemize}
\end{frame}


\begin{frame}
    \frametitle{Basic Block Sharing Illustrated}
        \begin{center}
            \pgfimage[height=6cm]{images/anti_re/basic_block_sharing_optimizations_2}
        \end{center}
\end{frame}

\begin{frame}
    \frametitle{Basic Block Sharing Problem Illustrated}
        \begin{center}
            \pgfimage[height=7cm]{images/anti_re/basic_block_sharing_optimizations_3}
        \end{center}
\end{frame}

\begin{frame}
    \frametitle{Basic Block Sharing. Alternative View}
        \begin{center}
            \pgfimage[height=6cm]{images/anti_re/basic_block_sharing_optimizations_4}
        \end{center}
\end{frame}




\begin{frame}
    \frametitle{More Disassembler Mucking}
    \begin{itemize}
        \item Junk instructions
            \pedbullet{Used to waste analysis time}
        \item Frequently encapsulated in blocks bordered with PUSHAD / POPAD
        \item Consider writing a script to replace PUSHAD.*POPAD with NOP instructions
    \end{itemize}
\end{frame}

\begin{frame}
    \frametitle{IDA Pro Vulnerability}
    \begin{itemize}
        \item Buffer Overflow Vulnerability
            \pedbullet{Buffer overflow in PE import directory parsing}
            \pedbullet{Easy to create an exploit for, however it breaks the loader}
            \pedbullet{There is a way around this, we know because Pedram wrote it ;-)}
        \item Patched in 4.7
            \pedbullet{Affects PEiD as well, which was also patched}
        \item Full advisory
            \pedbullet{http://www.idefense.com/application/poi/display?id=189}
    \end{itemize}
\end{frame}


\subsection{Anti-PE Analysis}
\begin{frame}
  \frametitle{Overview of Tricks}
  \begin{itemize}
    \item Invalid/malformed values in the header
    \item Misaligned sections, few applications correctly load them
    \item Relocation tricks
    \item TLS, running before the Entry Point
  \end{itemize}
\end{frame}

\begin{frame}
  \frametitle{Invalid Values}
  Malware can use invalid values in order to cause trouble to tools and hence, the analyst
  \begin{itemize}
    \item Uncommon ImageBase values
    \item Invalid data in LoaderFlags and NumberOfRvaAndSizes
    \item Large SizeOfRawDataValues \pedref{SOTM-33}
  \end{itemize}
\end{frame}

\begin{frame}
  \frametitle{Misaligned Sections}
  \begin{itemize}
    \item The Optional Header contains members describing:
      \begin{itemize}
        \item The file aligment (FileAligment)
        \item The memory alignment of the sections (SectionAlignment)
      \end{itemize}
    \item The section\'{}s starting file offset (PointerToRawData) is usually aligned to FileAligment
    \item However, it\'{}s possible to specify an unaligned offset. Windows will round it down to the largest aligned value smaller than the given offset
  \end{itemize}
\end{frame}

\begin{frame}
  \frametitle{Misaligned Sections}
  \begin{center}
    \includegraphics<1>[height=7cm]{images/anti_re/PE-Misaligned_Sections.pdf}%
  \end{center}
\end{frame}


\begin{frame}
  \frametitle{Relocation Tricks}
  \begin{itemize}
    \item Relocations are meant to provide for a mechanism to rebase an image
    \item Relocations are meant to patch references in the code pointing to locations that change if the image is rebased
    \item It is possible to use this to actually patch any data \pedref{VB2001}
    \item Abusing this technique allows an attacker to modify the image during load
  \end{itemize}
  \emph{skape} has an excellent write-up on the topic in Uninformed 6 \pedref{locreate}
\end{frame}


\begin{frame}
    \frametitle{Relocation of a DLL}
        \begin{center}
            \pgfimage[height=7cm]{images/anti_re/relocation_01}
        \end{center}
\end{frame}


\begin{frame}
    \frametitle{Relocation of a DLL}
        \begin{center}
            \pgfimage[height=7cm]{images/anti_re/relocation_02}
        \end{center}
\end{frame}


\begin{frame}
    \frametitle{Relocation of a DLL}
        \begin{center}
            \pgfimage[height=7cm]{images/anti_re/relocation_03}
        \end{center}
\end{frame}


\begin{frame}
    \frametitle{Relocation of a DLL}
        \begin{center}
            \pgfimage[height=7cm]{images/anti_re/relocation_04}
        \end{center}
\end{frame}


\begin{frame}
    \frametitle{Relocation of a DLL}
        \begin{center}
            \pgfimage[height=7cm]{images/anti_re/relocation_05}
        \end{center}
\end{frame}


\begin{frame}
    \frametitle{Relocation of a DLL}
        \begin{center}
            \pgfimage[height=7cm]{images/anti_re/relocation_06}
        \end{center}
\end{frame}



\subsection{Anti-VM}
\begin{frame}
    \frametitle{Virtual Machine Detection}
    \begin{itemize}
        \item Some malware have routines for detecting VMWare
        \item Malicious code that becomes aware that it is being executed in a virtual machine environment may behave completely differently then otherwise expected
        \item Facts like these keep reverse engineers employed
    \end{itemize}
\end{frame}


\begin{frame}
    \frametitle{Generic Detection}
    \begin{itemize}
        \item Timing
            \pedbullet{RDTSC}
            \pedbullet{Inconsistencies in the MMU might alter access times as compared to a real machine}
            \pedbullet{TLBs might also produce different memory access times}
            \pedbullet{External timers, NTPD}
        \item SIDT, SGDT (Redpill) to retrieve the base address of the IDT and GDT.
            \pedbullet{IDT value (\alert{hosts}): 0x80FFFFFF Windows, 0xC0FFFFFF Linux}
            \pedbullet{IDT value (\alert{guests}): 0xFFXXXXXX VMWare, 0xE8XXXXXX VirtualPC}
            \pedbullet{GDT value (\alert{hosts}): 0xC0XXXXXX}
            \pedbullet{The IDT test might fail on SMP systems. There's an IDT per processor}
    \end{itemize}
\end{frame}


\begin{frame}
    \frametitle{VMWare Detection}
    \begin{itemize}
        \item Trivial checks
            \pedbullet{Checking for VMWare Tools}
            \pedbullet{Checking for the service VMTools}
            \pedbullet{Spot the files in the filesystem, there are more than 50 files and folders with VMWare/vmx in their name}
            \pedbullet{In a guest VM with VMWare Tools installed there will be more than 300 references in the Windows registry and more than 1500 text strings in memory containing VMware}
    \end{itemize}
\end{frame}


\begin{frame}
    \frametitle{VMWare Detection}
    \begin{itemize}
        \item Hardware
            \pedbullet{MAC address of the network adaptor: 00-05-69, 00-0C-29 or 00-50-56}
            \pedbullet{Identifier of the graphics adaptor}
        \item Communication channel
            \pedbullet{IN with EAX=0x564D5868 "VMXh", EBX=parameters, ECX=command, EDX=5658h ("VX", port number)}
            \pedbullet{If the command is 0xA (request VMWare's verion) EBX will contain 0x564D5868 "VMXh" if it's a guest system }
        \item There are ways of working around some of these
    \end{itemize}
\end{frame}


\begin{frame}
    \frametitle{VirtualPC Detection}
    \begin{itemize}
        \item Device detection
        \item Communication channel
            \pedbullet{VirtualPC uses invalid instructions to communicate}
            \pedbullet{0F 3F x1 x2}
            \pedbullet{0F C7 C8 y1 y2}
        \item Instructions longer than 15 bytes are invalid but VirtualPC does not raise the corresponding exception
        \item CPUID returns "ConnetixCPU"
    \end{itemize}
\end{frame}


\begin{frame}
    \frametitle{Other VMs}
    \begin{itemize}
        \item Peter Ferrie wrote a detailed paper on attacks to all major virtual machine applications
            \pedbullet{\pedref{VMDetection4}}
        \item Also of interest are
            \pedbullet{\pedref{VMDetection1}}
            \pedbullet{\pedref{VMDetection2}}
            \pedbullet{\pedref{VMDetection3}}
    \end{itemize}
\end{frame}



\begin{frame}
    \frametitle{Exercises}
    \begin{itemize}
        \item Defeat IsDebuggerPresent() check using inline patching
        \item Verify that VMWare detection works
            \pedbullet{Check VM/Settings/Options/Advanced/DisableAcceleration and try again}
        \item Help IDA disassemble the ASPack decode routine with x86-emu.
        \item Get 0x90.exe to load in IDA
        \item At your leisure review the rest of the ridiculous protection mechanisms presented and dissected in:
            \pedbullet{http://project.honeynet.org/scans/scan33/}
    \end{itemize}
\end{frame}

\begin{frame}
  \frametitle{Solution: PE Optional Header Trickery}
  \begin{itemize}
    \item Binary from \emph{Honeynet Project} Scan of The Month 33
    \item \emph{ImageBase} normally 0x00400000
    \item Modification is simply a nuisance
    \item \emph{NumberOfRvaAndSizes} normally 0x00000010
  \end{itemize}
  \begin{center}
    \pgfimage[width=11cm]{images/anti_re/bad_pe_header_values}
  \end{center}
\end{frame}

\begin{frame}
  \frametitle{Solution: PE Optional Header Trickery}
  \begin{itemize}
    \item \emph{LoaderFlags} normally \emph{NULL}
      \begin{center}
        \pgfimage[width=11cm]{images/anti_re/loaderflags}
      \end{center}
    \item Debugger vulnerabilities discovered by \emph{Nicolas Brulez}
      \begin{itemize}
        \item \emph{NumberOfRvaAndSizes} modification crashes \emph{SoftICE}
        \item \emph{LoaderFlags} + \emph{NumberOfRvaAndSizes} modifications crashes \emph{OllyDBG}
      \end{itemize}
    \item To fix
      \begin{itemize}
        \item Set \emph{NumberOfRvaAndSizes} to \emph{0x00000010}
        \item Optionally set \emph{LoadFlags} to \emph{0x00000000}
      \end{itemize}
  \end{itemize}
\end{frame}

\begin{frame}
  \frametitle{Solution: PE Section Header Trickery}
  \begin{itemize}
    \item Large \emph{SizeOfRawData}
    \item Causes many tools to crawl or crash
      \begin{itemize}
        \item Ex: \emph{IDA Pro} will attempt to allocate massive memory chunk
      \end{itemize}
      \begin{center}
        \pgfimage[width=11cm]{images/anti_re/nicolasb_section}
      \end{center}
    \item To fix
      \begin{itemize}
        \item Set NicolasB section raw size to \emph{0x00000000}
      \end{itemize}
    \item Note: virtual size == raw size, binary is not compressed
  \end{itemize}
\end{frame}
